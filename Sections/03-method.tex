%*----------- SLIDE -------------------------------------------------------------
\begin{frame}[t]{Metodologia - Avaliação do modelo}
    Para analisar o modelo criado, os autores adotadaram quatro métricas 
    \vspace{0.4cm}

    O parâmetro \textbf{\textit{"Accuracy"}}  representa a fração das predições feitas corretamente
    \vspace{0.4cm}

    A \textbf{\textit{"Precision"}} é a razão entre as predições verdadeiras corretas dentre todas as predições verdadeiras, corretas ou não 
    \begin{table}[]
        \vspace{0.8cm}
        \centering
        \begin{tabular}{ll}
            \begin{math} Accuracy = \frac{Number\,of\,correct\,predictions}{Total\,number\,of\,predictions} \end{math} 
            &
            \begin{math} Precision = \frac{True\,Positive}{True\,Positive + False\,Negative} \end{math} 
        \end{tabular}
    \end{table}
    
%*----------- notes
    \note[item]{Notes can help you to remember important information. Turn on the notes option.}
\end{frame}

%*----------- SLIDE -------------------------------------------------------------
\begin{frame}[t]{Metodologia - Avaliação do modelo}
    O parâmetro \textbf{\textit{"Recall"}} é a habilidade do modelo de encontrar os verdadeiros positivos
    \vspace{0.4cm}
    
    A \pmb{"\begin{math} F_1Score \end{math}"} é a média harmônica pondera entre os parâmetros \textit{"Precision"} e \textit{"Recall"}
    \begin{table}[]
        \vspace{0.8cm}
        \centering
        \begin{tabular}{ll}
            \begin{math} Recall = \frac{True\,Positive}{True\,Positive + False\,Negative} \end{math}
            &
            \begin{math} F_1Score = 2\cdot\frac{Precision \cdot Recall}{Precision + Recall} \end{math}
        \end{tabular}
    \end{table}
    
%*----------- notes
    \note[item]{Notes can help you to remember important information. Turn on the notes option.}
\end{frame}
% FALTA: contexto(onde e como), mapa, solução